%---------------------------------------------------------------------------%
%->> Frontmatter
%---------------------------------------------------------------------------%
%-
%-> 生成封面
%-
\maketitle% 生成中文封面
\MAKETITLE% 生成英文封面
%-
%-> 作者声明
%-
\makedeclaration% 生成声明页
%-
%-> 中文摘要
%-
\intobmk\chapter*{摘\quad 要}% 显示在书签但不显示在目录
\setcounter{page}{1}% 开始页码
\pagenumbering{Roman}% 页码符号

增强现实是一种将虚拟信息通过虚实融合技术叠加显示到真实场景恰当位置处,以此提升用户与真实环境之间的交互体验的技术,它
广泛应用于科普教育、军事仿真、文化旅游、医养健康、装备维保等众多领域。虚实融合技术是增强现实领域的关键技术之一,
它通过跟踪相机自身位姿,将虚拟资料在环境中准确注册,直接影响了系统的真实性体验。
基于特征的视觉SLAM技术能一定程度上解决相机位姿估计和环境几何信息理解的问题,从而帮助虚实融合,
但在动态环境下的位姿估计和建图性能、语义信息表达能力等方面仍有不足。本文针对以上问题,构建了结合深度神经网络和视觉SLAM框架的语义SLAM系统,
对于对动态场景下位姿估计优化方法、语义地图的构建和表示方法及在增强现实系统中的应用方法进行了研究。本文的主要工作和贡献如下:
{
\setlist[enumerate]{}% restore default behavior
\begin{enumerate}[nosep]
    \item 针对动态场景下基于特征的视觉SLAM方法相机位姿估计精度低的问题,提出了双阶段的动态外点滤除算法。此算法结合基于语义类别的
    自适应权重生成方法和基于极线约束的动态一致性检测方法,能够对于不同类别的物体自适应生成不同的判定条件,判断其是否处于运动状态,
    并移除处于运动状态的物体的轮廓内的特征点。此算法在TUM RGB-D数据集上进行了验证,在动态场景下其结果比ORB-SLAM2有数量级式的提升,
    提升幅度在大部分指标上优于业界先进系统,表明它能够有效提升基于特征的视觉SLAM系统在动态场景下的位姿估计精度和鲁棒性。
    \item 针对现有方法缺乏对环境高层语义理解导致虚实融合真实感不强的问题,分别探究了基于局部特征的预置语义库地图和语义SLAM系统生成的
    物体级三维语义点云地图的构建和表示方法。对于预置语义库地图提出了特征识别和跟踪方法,验证了其对室外环境的表示能力和在增强现实导航系统中的应用能力。
    对于物体级三维语义点云地图,本文提出了针对语义类别的物体跟踪策略,能够对环境中的物体分别建模、跟踪,适应动态变化的环境,并进行语义关联。
    此方法在动态环境和静态环境下分别进行了验证,结果表明它能够提高建图的鲁棒性,建立随环境变化而变化的物体级语义地图,从而提高对三维环境的表示能力。
    \item 增强现实原型系统的设计与实现\\这部分研究工作设计并实现了一个增强现实原型系统,融合了述研究的算法,在虚实融合阶段进行了
    优化,提升了系统的真实感体验。
\end{enumerate}
}

\keywords{增强现实,语义SLAM,虚实融合,动态环境,语义地图}% 中文关键词
%-
%-> 英文摘要
%-
\intobmk\chapter*{Abstract}% 显示在书签但不显示在目录

Abstract

\KEYWORDS{Abstract}% 英文关键词
%---------------------------------------------------------------------------%
