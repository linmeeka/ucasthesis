\chapter{总结与展望}\label{chap:6}
\section{研究总结}
增强现实是一种当今社会中高速发展的新兴技术,丰富多彩增强现实应用正在逐渐进入人们的娱乐、学习、生产等领域。虚实融合技术是增强现实
领域的关键技术之一,它通过跟踪相机自身位姿,将虚拟资料注册在显示环境中的恰当位置。虚实融合效果的好坏能够影响增强现实系统在
物理遮挡问题等方面的表现,从而影响系统的真实性体验。基于特征的视觉SLAM技术能一定程度上解决相机位姿估计和环境几何信息理解的问题,从而帮助虚实融合,
但在动态环境下的位姿估计和建图、语义信息表达等方面仍有不足。本文针对以上问题,构建了结合深度神经网络和视觉SLAM框架的语义SLAM系统,
对于对动态场景下位姿估计优化方法、语义地图的构建和表示方法及在增强现实系统中的应用方法进行了研究。本文的主要工作和贡献如下:
{
\setlist[enumerate]{}% restore default behavior
\begin{enumerate}[nosep]
    \item 动态场景下基于语义信息的位姿估计优化方法\\这部分研究工作针对动态场景,提出了双阶段的动态外点滤除算法,此算法结合基于语义类别的
    自适应权重生成方法和基于极线约束的动态一致性检测方法,能够对于不同类别的物体自适应生成不同的判定条件,判断其是否处于运动状态,
    并移除处于运动状态的物体的轮廓内的特征点。此算法在TUM RGB-D数据集上进行了验证,在动态场景下其结果比ORB-SLAM2有数量级式的提升,
    提升幅度在大部分指标上优于业界先进系统,表明它能够有效减少环境中运动物体对于SLAM系统位姿估计的干扰,提升基于特征的视觉SLAM系统在动态场景下的位姿估计精度和鲁棒性。
    \item 语义地图的构建和表示方法\\这部分研究工作针对不同层级语义地图对环境的表示能力,分别探究了基于局部特征的预置语义库地图和语义SLAM系统生成的
    物体级三维语义点云地图的构建和表示方法。对于预置语义库地图提出了特征识别和跟踪方法,验证了其对室外环境的表示能力和在增强现实导航系统中的应用能力。
    对于物体级三维语义点云地图,本文提出了针对语义类别的物体跟踪策略,能够对环境中的物体分别建模、跟踪,适应动态变化的环境,并进行语义关联。
    此方法在动态环境和静态环境下分别进行了验证,结果表明它能够提高建图的鲁棒性,建立随环境变化而变化的物体级语义地图,从而提高对三维环境的表示能力。
    \item 增强现实原型系统的设计与实现\\这部分研究工作设计并实现了一个增强现实原型系统,融合了述研究的算法,在虚实融合阶段进行了
    优化,提升了系统的真实感体验。
\end{enumerate}
}

\section{未来展望}
增强现实技术是当前最有研究前景和应用价值的技术之一,各大高校、公司和研究机构都有相关的研究战略部署,如微软公司发布的HoloLens平台、
苹果公司发布的ARKit平台、谷歌公司发布的ARCore平台、高通公司发布的Vuforia平台等,都取得了不俗的表现,也拥有很大的发展空间。
本文研究的基于语义SLAM的增强现实虚实融合关键技术,虽然一定程度上提高了系统在动态环境下位姿估计精度,建立了物体级的三维语义地图,
但仍存在一些局限,仍需在未来工作中探究。具体的未来研究方向有:
{
\setlist[enumerate]{}% restore default behavior
\begin{enumerate}[nosep]
    \item 系统实时性较低\\当前系统受限于语义分割模块对图像的处理速度过慢,无法取得实时运行的效果,这限制了基于此系统的增强现实
    应用在实际环境中的应用。在未来的工作中,可以尝试使用更高效的实例分割网络结构,或按照一定策略间隔取帧进行实例分割,以取得
    位姿估计精度和运行效率的平衡。
    \item 静态环境下位姿估计提升效果不显著\\当前系统通过检测场景中的动态目标,滤除其轮廓内的特征点的方法,来减少动态环境对
    相机位姿估计的影响,但这种方法在静态环境或低动态环境下对位姿估计的提升效果并不显著。其原因之一是当前基于光流的动态一致性
    检测方法对光照等环境变化敏感,容易造成误判,经特征点滤除之后,图像中用于迭代、优化的特征点总数减少。在未来的工作中,
    可以尝试使用其他物体移动的检测方法,以更精确地滤除动态特征点。
    \item 模型精度不足\\当前语义建图模块所建的物体级语义地图的模型仍比较粗糙,存在边缘模糊、部分缺失的情况。主要原因是实例分割
    模块输出的语义mask边缘不够精细,在发生物理遮挡、视野部分丢失的情况时,也会导致建模不完整。在未来的工作中,可以尝试结合
    基于几何特性的分割方法,在二维图像或三维点云上进行物体边缘分割,以细化语义分割模块实例分割的结果。还可以使用能够支持物体形变
    的建模方法,在更长的时间跨度上对模型进行跟踪和维护。
    
\end{enumerate}
}
